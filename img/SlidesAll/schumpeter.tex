\documentclass[10pt]{article}
%%%%%%%%%%%%%%%%%%%%%%%%%%%%%%%%%%%%%%%%
\usepackage{amsmath}
\usepackage{verbatim}
\usepackage[usenames,dvipsnames]{color}
\usepackage{ulem}
\usepackage{setspace}
\usepackage{lscape}
\usepackage{longtable}
\usepackage[top=1.25in,bottom=1.5in,left=1in,right=1.5in,landscape]{geometry}
\usepackage{graphicx}
\usepackage{epstopdf}
\usepackage[usenames,dvipsnames]{pstricks}
\usepackage{epsfig}
\usepackage{pstricks-add}
\usepackage{pst-node}
\usepackage{fancyhdr}
\usepackage[absolute,showboxes]{textpos}

%TCIDATA{OutputFilter=LATEX.DLL}
%TCIDATA{Version=5.00.0.2552}
%TCIDATA{<META NAME="SaveForMode" CONTENT="1">}
%TCIDATA{Created=Thursday, August 28, 2003 13:38:44}
%TCIDATA{LastRevised=Thursday, August 14, 2008 15:20:27}
%TCIDATA{<META NAME="GraphicsSave" CONTENT="32">}
%TCIDATA{<META NAME="DocumentShell" CONTENT="Standard LaTeX\Blank - Standard LaTeX Article">}
%TCIDATA{Language=American English}
%TCIDATA{CSTFile=LaTeX article (bright).cst}

\setcounter{MaxMatrixCols}{10}

\newenvironment{proof}[1][Proof]{\noindent\textbf{#1.} }{\ \rule{0.5em}{0.5em}}
\setlength{\columnsep}{.2in}

\renewcommand{\labelitemii}{$\cdot$}

\pagestyle{fancy} \fancyhead{} \fancyfoot{} \rfoot{} \lfoot{}

\newcommand{\slide}[2]{
\begin{textblock}{11}(0,0)
\textcolor{Black}{\textbf{\huge \rule{0pt}{1in} \raisebox{.2in}{#1}}}
\end{textblock}
\begin{Large} \noindent
#2
\end{Large}
\vfill \pagebreak}

\setlength{\TPHorizModule}{1in}
\setlength{\TPVertModule}{1in}
\textblockcolour{Yellow}
\renewcommand{\headrulewidth}{0pt}



\begin{document}
\onehalfspacing 

\lfoot{The Schumpeterian Model} \rfoot{Economic Growth}

\slide{Creative Desctruction}{The Romer model thought of innovation as adding new intermediate goods to the market (cars, robots, iPhones), but once invented, each good was fixed in quality.

\vspace{.25in}\noindent The Schumpeterian model conceives of innovation as improving existing products (iPhone 6 vs. iPhone 5 vs. iPhone 4), with newer versions replacing old versions. Hence the ``creative destruction''. 

\vspace{.25in}\noindent Schumpeterian growth changes the mechanical details of growth, but not the general conclusions
\begin{itemize}
	\item The long-run trend growth rate depends on population growth
	\item The allocation of workers to research may not be optimal
\end{itemize}

\vspace{.25in}\noindent One advantage of the Schumpeterian model is that it explicitly allows us to think about firm dynamics, or the creation and destruction of firms over time.
}

\slide{Mechanics of Growth}{Final goods are produced using
\begin{equation}
Y = K^{\alpha}(A_i L_Y)^{1-\alpha}
\end{equation}
where $A_i$ is the productivity of the latest version of technology. So $A_2 > A_1$, and $A_{100} > A_{99}$. 

\vspace{.25in}\noindent $A_i$ moves up in discrete jumps, so
\begin{equation}
A_{i+1} = (1+\gamma)A_i
\end{equation}
where $\gamma$ is the ``step size'', or how much productivity rises each time we innovate.

\vspace{.25in}\noindent Growth occurs when we innovate, but that doesn't always happen. The growth rate of $A_i$ from \textit{innovation} to \textit{innovation} is
\begin{equation}
\frac{A_{i+1}-A_i}{A_i} = \gamma
\end{equation}
but this is not how fast $A_i$ grows over \textit{time}.
}

\slide{The Speed of Innovation}{The chance that any given researcher will produce an innovation at any given moment is
\begin{equation}
\overline{\mu} = \theta \frac{L_A^{\lambda-1}}{A_i^{1-\phi}}
\end{equation}
or the probability of innovating depends on the same forces as before: duplication due to other researchers and the spillovers of $A_i$ on research.

\vspace{.25in}\noindent For the economy as a whole, the probability of making an innovation depends on how many researchers are working, so
\begin{equation}
P(Innovation) = \overline{\mu}L_A = \theta \frac{L_A^{\lambda}}{A_i^{1-\phi}}.
\end{equation}
}

\slide{Long-run Growth Rate}{The growth rate of this economy is not immediately obvious. 
\begin{itemize}
	\item At any given point in time, no one may have innovated, so growth is zero. 
	\item When someone does innovate, $A_i$ jumps by $\gamma$, so growth is very rapid in that moment
	\item Look at average growth over long period, smoothing this out
\end{itemize}
\begin{center}
\scalebox{1}
{
\begin{pspicture}(10,9)
\psline{->}(0,0)(9,0) \psline{->}(0,0)(0,8) \rput(9.3,-.3){TIME} \rput(0,8.3){LOG $y$}
\psline[linecolor=gray,linewidth=2pt,linestyle=dashed,dash=3pt 2pt](.2,1)(8.5,6.3) 
\psline[linewidth=2pt](.2,1)(.5,1)(.5,1.5)(1.5,1.5)(1.5,2)(1.9,2)(1.9,2.5)(3,2.5)(3,2.8)(3.5,2.8)(3.5,3.3)(3.8,3.3)(3.8,3.8)(5,3.8)(5,4.3)(5.5,4.3)(5.5,4.8)(6.2,4.8)(6.2,5.3)(7.4,5.3)(7.4,5.8)(8,5.8)(8,6.3)
\rput(10.5,6.3){\small Balanced growth path} \rput(5,5.4){\small Actual growth}
\end{pspicture}
}
\end{center} 

}

\slide{Expected Growth Rate}{So the expected growth rate along the BGP (the smoothed line) is
\begin{equation}
E\left[\frac{\dot{A}}{A}\right] = \gamma \overline{\mu} L_A = \gamma \theta \frac{L_A^{\lambda}}{A_i^{1-\phi}}.
\end{equation}
\begin{itemize}
	\item $\gamma$ tells us how much $A$ jumps when an innovation occurs
	\item $\overline{\mu}L_A$ tells us the expected value of the number of jumps
\end{itemize}

\vspace{.25in}\noindent Using this, what is the expected growth rate of $A$ along the BGP? Along the BGP the expected growth rate of $A$ will be constant. Using the above and taking logs and derivatives
\begin{equation}
0 = \lambda \frac{\dot{L_A}}{L_A} - (1-\phi)E\left[\frac{\dot{A}}{A}\right] 
\end{equation}
which given that $\dot{L_A}/L_A = n$ along the BGP, means that
\begin{equation}
E\left[\frac{\dot{A}}{A}\right] = g = \frac{\lambda}{1-\phi}n
\end{equation}
which is identical to what we got in the Romer model.
}

\slide{Comparison}{The Schumpeterian model doesn't change our conclusion about the long-run trend growth rate.
\begin{itemize}
	\item Due to assumption that technological change depends on $L_A^{\lambda}$ and $A^{1-\phi}$
	\item Not relevant whether $A$ is more goods or better goods
\end{itemize}

\vspace{.25in}\noindent Note that $\gamma$ does not feature in the long-run growth rate
\begin{itemize}
	\item Larger $\gamma$ boosts the size of jumps in $A$, which would be good for growth
	\item Larger $\gamma$, though, raises $A$, making it harder to find the next innovation
	\item These effects cancel out
\end{itemize}

\vspace{.25in}\noindent Schumpeterian model differs in the underlying economics, and will differ in the equilibrium value of $s_R$
}

\slide{Final Goods Production}{Final goods produced using
\begin{equation}
Y = L_Y^{1-\alpha}A_i^{1-\alpha}x_i^{\alpha}
\end{equation}
where $x_i$ is a single intermediate (or capital) good used in the final goods sector. It is indexed by $i$ because each intermediate good has a specific productivity level, $A_i$ associated with it. 

\vspace{.25in}\noindent Similar to before, final good firms will maximize profits, 
\begin{itemize}
	\item Choose how many units of $x_i$ to use
	\item Choose which version of $x_i$ to use (latest, or older less productive version)
	\item Will turn out that all versions cost the same, so they will pick best one
\end{itemize}

}

\slide{Final Good Sector Profits}{They will
\begin{equation}
max_{L_Y,x_i} L_Y^{1-\alpha}A_i^{1-\alpha}x_i^{\alpha} - w L_Y - p_i x_i
\end{equation}
giving first-order conditions of
\begin{eqnarray}
w &=& (1-\alpha)\frac{Y}{L_Y} \\
p_i &=& \alpha L_Y^{1-\alpha}A_i^{1-\alpha}x_i^{\alpha-1} 
\end{eqnarray}
which again is just that the firm sets marginal cost equal to marginal product.

\vspace{.25in}\noindent As in the Romer model, the elasticity of demand for the intermediate good is $\alpha-1$
}

\slide{Intermediate Good Firms}{Int. good firms are monopolists at producing their version of the int. good. As before, they transform one unit of capital into one unit of the int. good. Their profits are
\begin{equation}
\pi_i = p_i(x_i) x_i - r x_i
\end{equation}
The first-order condition is 
\begin{equation}
p'_i(x_i)x_i + p_i(x_i) = r
\end{equation}
or set marginal revenue to marginal cost.

\vspace{.25in}\noindent As before, we can transform this FOC into
\begin{equation}
p_i = \frac{1}{1+\frac{p'_i(x_i) x_i}{p_i}}r
\end{equation}
which given the elasticity we found for final goods firm demand gives
\begin{equation}
p_i = \frac{r}{\alpha}
\end{equation}

}

\slide{Markup Pricing}{Similar to the Romer model, int. good firms charge a markup over marginal cost.
\begin{itemize}
	\item This generates the profits that will motivate innovation
	\item The price they charge does \textit{not} depend on the version of the int. good they produce
	\item So all versions of $x_i$ sell for the same price, final good firms only buy the best one (highest $A_i$)
\end{itemize}

\vspace{.25in}\noindent This set-up ensures the creative desctruction in the economy. Once a new innovation occurs (a new $x_{i+1}$) with a higher productivity, the old intermediate good ($x_i$) firm goes out of business and the new firm takes over completely.

}

\slide{Aggregate Output}{Given that only one int. good firm operates at a time, it must be that
\begin{equation}
x_i = K
\end{equation}
meaning that aggregate output is
\begin{equation}
Y = K^{\alpha} (A_i L_Y)^{1-\alpha}
\end{equation}
which is the same as the standard production function we always use.

\vspace{.25in}\noindent We can solve for the distribution of income as we did in the Romer model, yielding
\begin{eqnarray}
wL &=& (1-\alpha)Y \\
rK &=& \alpha^2 Y \\
\pi_i &=& \alpha(1-\alpha)Y
\end{eqnarray}
with the only difference being that all profits accrue to the one operating int. good firm. 
}

\slide{Research Sector}{Again, we want to understand the incentive to do research. Inventing a new version $x_{i+1}$ gives you a patent on that good you can sell (or use to be the monopolist). Use arbitrage to value the patent
\begin{equation}
r P_A = \pi + \dot{P}_A - (\overline{\mu}L_A) P_A
\end{equation}
\begin{itemize}
	\item $rP_A$ is again the value of putting your money in the bank instead
	\item $\pi + \dot{P}_A$ is the value of the patent: profits plus capital gains
	\item $(\overline{\mu}L_A) P_A$ captures the fact that with probability $\overline{\mu}L_A$, you will be replaced as the monopolist by the next innovator, so it is a negative.
\end{itemize}

\vspace{.25in}\noindent Re-arrange to
\begin{equation}
r = \frac{\pi}{P_A} + \frac{\dot{P}_A}{P_A} - \overline{\mu}L_A
\end{equation}
and for convenience let $\mu =  \overline{\mu}L_A$ so
\begin{equation}
r = \frac{\pi}{P_A} + \frac{\dot{P}_A}{P_A} - \mu
\end{equation}
}

\slide{Patents Along BGP}{As in the Romer model, we want to consider the value of a patent along the BGP, where $r$ is constant. This implies $\pi$ and $P_A$ must grow at the same rate. 
\begin{itemize}
	\item We know that profits are $\pi = \alpha(1-\alpha)Y$, so profits grow at the rate $g + n$
	\item We know that $g = \gamma \overline{\mu}L_A = \gamma \mu$ along the BGP
\end{itemize} 	
so we have that
\begin{equation}
r = \frac{\pi}{P_A} + \gamma\mu + n - \mu
\end{equation}
which solves to
\begin{equation}
P_A = \frac{\pi}{r-n + \mu(1-\gamma)}.
\end{equation}

\vspace{.25in}\noindent Again, patents are the present discounted value of profits. Now the discount rate is higher because of $\mu$, which captures the chance of being replaced.

\vspace{.25in}\noindent Note that no existing monopolist would ever buy the new patent. Why? Because they have to sacrifice their existing profits, meaning they will not pay as much for the new patent. So always new firms coming into existence. ``Arrow Replacement Effect''.
}

\slide{Equilibrium for Labor}{Again, individuals can either do research to get the next idea, or work in the final goods sector. They move back and forth until the returns to these two activities are identical, or
\begin{equation}
(1-\alpha)\frac{Y}{L_Y} = \overline{\mu}P_A
\end{equation}
where $\overline{\mu}$ is the chance that an individual will innovate, and $P_A$ is the value of that innvotion to them.
\begin{eqnarray}
(1-\alpha)\frac{Y}{L_Y} &=& \overline{\mu}\frac{\pi}{r-n + \mu(1-\gamma)} \\
(1-\alpha)\frac{Y}{L_Y} &=& \overline{\mu}\frac{\alpha(1-\alpha)Y}{r-n + \mu(1-\gamma)} \\
\frac{1}{L_Y} &=& \frac{\mu}{L_A}\frac{\alpha}{r-n + \mu(1-\gamma)} \\
\frac{L_A}{L_Y} &=& \mu \frac{\alpha}{r-n + \mu(1-\gamma)} \\
\frac{s_R}{1-s_R} &=& \mu \frac{\alpha}{r-n + \mu(1-\gamma)}
\end{eqnarray}
which solves to
\begin{equation}
s_R = \frac{1}{1 + \frac{r-n+\mu(1-\gamma)}{\alpha \mu}}.
\end{equation}
}

\slide{Equilibrium $s_R$}{Found
\begin{equation}
s_R = \frac{1}{1 + \frac{r-n+\mu(1-\gamma)}{\alpha \mu}}
\end{equation}
\begin{itemize}
	\item Same discount factor $r-n$. If that goes up, value of patents goes down, lower $s_R$
	\item First effect of $\mu$: from $\mu(1-\gamma)$ captures the fact that as the probability of innovation goes up, the value of patents declines due to replacement effects
	\item Second effect of $\mu$: from $\alpha \mu$ captures the fact that as the probability of innovation goes up, you are more likely to get a patent in th first place
	\item On net, the second effect ``wins''. You get a patent now, and will only be replace later, so if $\mu$ goes up, $s_R$ goes up
\end{itemize}
}

\slide{Comparing Schumpeter and Romer}{The long-run growth rate is identical at $g = \lambda n/(1-\phi)$. The difference is in the \textit{level} of income along the BGP implied by the value of $s_R$.

\begin{itemize}
	\item Schumpeterian model has higher $s_R$ if $g < r-n$. In this case the discount rate is very large, and so I care most about profits in the immediate future and little about the fact that I might be replaced some day. So more people do research than in Romer.
	\item Schumpeterian model has lower $s_R$ if $g > r-n$. In this case the discount rate is low, so people do care about the future replacement a lot. Hence $s_R$ is low compared to Romer.
\end{itemize}

\vspace{.25in}\noindent Remember that higher $s_R$ is not necessarily optimal. $y(t)$ along the BGP depends both positively and negatively on $s_R$. There is no sense in which Romer or Schumpeter is ``better''. They are different ways of conceiving of the growth process.

}

\end{document}